\documentclass{beamer}
\usepackage{tikz}
\usepackage{amssymb}
\usepackage{graphicx}
\usepackage{subcaption}
\usepackage{adjustbox}
\usepackage{array}
\usepackage{algorithm2e}
\usepackage{lmodern}
\usepackage{caption}
\renewcommand{\figurename}{}
\usepackage{xcolor}
\definecolor{myblue}{rgb}{0.2,0.5,0.8} % Ajustez les valeurs RGB pour obtenir la teinte désirée
\definecolor{mygreen}{RGB}{34,139,34} % Un vert pour alpha
\definecolor{mypurple}{RGB}{148,0,211} % Un violet pour k
\definecolor{myorange}{RGB}{255,165,0} % Un orange pour f
\newcommand{\blue}[1]{{\color{myblue}#1}}
%\newcommand{\blue}[1]{{\color{blue}#1}}
\newcommand{\red}[1]{{\color{red}#1}}
\newcommand{\green}[1]{{\color{mygreen}#1}}
\newcommand{\purple}[1]{{\color{mypurple}#1}}
\newcommand{\orange}[1]{{\color{myorange}#1}}
\usepackage[
  backend=bibtex,
  style=authoryear,
  maxcitenames=1, 
  maxbibnames=999
]{biblatex}
\addbibresource{mybibfile.bib}


\usetikzlibrary{fadings}
\usetikzlibrary{shadows}
\usetikzlibrary{calc}
\usetheme{Antibes}
\usecolortheme{lily}
\definecolor{darkgreen}{rgb}{0,0.5,0}
\definecolor{darkred}{rgb}{0.5,0,0}
\DeclareMathOperator*{\argmin}{arg\,min}
% New column type for centering content in a table cell
% Define a new column type "C" that centers content both vertically and horizontally
\newcolumntype{M}[1]{>{\centering\arraybackslash}m{#1}}

\setbeamertemplate{footline}
{
  \leavevmode%
  \hbox{%
  \begin{beamercolorbox}[wd=.33\paperwidth,ht=2.25ex,dp=1ex,center]{author in head/foot}%
    \usebeamerfont{author in head/foot}\insertshortauthor
  \end{beamercolorbox}%
  \begin{beamercolorbox}[wd=.33\paperwidth,ht=2.25ex,dp=1ex,center]{title in head/foot}%
    \usebeamerfont{title in head/foot}\insertshortinstitute
  \end{beamercolorbox}%
  \begin{beamercolorbox}[wd=.33\paperwidth,ht=2.25ex,dp=1ex,center]{date in head/foot}%
    \usebeamerfont{date in head/foot}\insertshortdate{}\hspace*{2em}\insertframenumber/\inserttotalframenumber\hspace*{2ex}
  \end{beamercolorbox}}%
  \vskip0pt%
}
% Réduire la taille de la police de la Plan de la présentation
\setbeamerfont{section in toc}{size=\small}
\setbeamerfont{subsection in toc}{size=\small}

\title{Génération de maillages hexaédriques pour des simulations de grandes déformations}
\author[David Desobry]{\texorpdfstring {\scriptsize \textit{Soutenance de thèse \\ \vspace{0.1cm} }} \texorpdfstring{\scriptsize par \\ \vspace{0.1cm}} \texorpdfstring{\normalsize David DESOBRY {}}{}}
\institute[Université de Lorraine] % Your institution as it will appear on the bottom of every slide, may be shorthand to save space
{

{\small \textit{Directeurs : }}
\medskip
{\small Dmitry SOKOLOV, Nicolas RAY, Jeanne PELLERIN}
\vspace{0.2 cm}
\texorpdfstring{\\ \scriptsize \textit{Jury : }}
\medskip
{\small \ \ \ Christian GENTIL, Jeanne PELLERIN,\quad \quad \quad \quad \quad \\  
\quad Simon CALDERAN, Emmanuel JEANDEL\\}
\begin{center}
    \noindent
    \begin{minipage}{.195\textwidth}
        \centering
    \includegraphics[width=.8\linewidth]{img/new_images/inria.jpg}
    \end{minipage}%
    \begin{minipage}{.195\textwidth}
        \centering
        \includegraphics[width=.6\linewidth]{img/new_images/loria.png}
    \end{minipage}
    \begin{minipage}{.195\textwidth}
    \centering
    \includegraphics[width=.8\linewidth]{img/new_images/UL.png}
    \end{minipage}%
    \begin{minipage}{.195\textwidth}
        \centering
    \includegraphics[width=.6\linewidth]{img/new_images/total_energies.jpg}
    \end{minipage}
    \begin{minipage}{.195\textwidth}
        \includegraphics[width=.6\linewidth]{img/new_images/hutchinson.png}
    \end{minipage}
\end{center}
}
\date{23 Août 2023}

\begin{document}

\frame{\titlepage}

\begin{frame}{Partenariat entre Hutchinson et l'équipe PIXEL}
    \centering
    
    \only<1>{
        \textbf{Hutchinson : Filiale de TotalEnergies} \\
        \begin{itemize}
            \item Spécialiste de la simulation de grande déformation hyper-élastique des élastomères. 
            \item Propriétaire du logiciel de simulation Numea :
        \end{itemize}
        \includegraphics[width=.8\linewidth]{img/introduction/hutchinson_sapin.PNG}
    }
    
    \only<2>{
        \textbf{Équipe PIXEL : Inria Nancy Grand-Est} \\
        \begin{itemize}
            \item Spécialiste des méthodes de paramétrisation globale.
        \end{itemize}
        \includegraphics[width=\linewidth]{img/cubecover/pipeline.PNG}
    }
\end{frame}



\begin{frame}
    \frametitle{Plan de la présentation}
    \tableofcontents[currentsection, sectionstyle=show/show, subsectionstyle=show/hide/hide]
\end{frame}

\section{État de l'art : Simulation de déformation et qualité d'un maillage}

\subsection{Introduction aux simulations numériques et aux maillages}
\begin{frame}
    \frametitle{Plan de la présentation}
    \tableofcontents[currentsubsection, sectionstyle=show/shaded, subsectionstyle=show/shaded/hide]
\end{frame}
\begin{frame}{Evolution de la roue}
    \begin{figure}
        \centering
        \includegraphics[width=0.79\textwidth]{img/plagiat/evolution-roue.jpg}
        \caption{
        L'évolution de la roue depuis son invention -3500 avant J.C (gauche), son allègement avec les inventions des roues à rayons en -2000 avant J.C (milieu), jusqu'aux roues à pneus en caoutchouc des véhicules que l'on utilise aujourd'hui (droite).
        }
        \label{fig:invention_roue}
    \end{figure}
\end{frame}
\begin{frame}{Simulation d'aquaplanage par Michelin}
    \begin{figure}
        \centering
        \includegraphics[width=0.79\textwidth]{img/plagiat/simulation_pneumatique_michelin.PNG}
        \caption{
            L'objectif est de s'assurer que les pneus développés par l'entreprise respectent les normes de sécurité en termes d'aquaplanage, 
            même lorsqu'ils sont usés. Image issue de \url{https://www.youtube.com/watch?v=S6csZMRy_xk}
        }
        \label{fig:simu_michelin}
    \end{figure}
\end{frame}

\iffalse
\begin{frame}{Simulation Numérique: Choix de la méthode}

    \begin{itemize}
        \item \only<1-5>{\textbf{Le modèle physique} \textcolor{darkgreen}{\footnotesize(hyperélasticité, plasticité, rupture)}}\only<6->{\textbf{Le modèle physique:} \textcolor{darkred}{\footnotesize{Choix: hyperélasticité}}}
        \item \only<2-5>{\textbf{La méthode numérique} \textcolor{darkgreen}{\footnotesize(méthode des éléments finies, différences finies, volumes finis)}}\only<6->{\textbf{La méthode numérique:} \textcolor{darkred}{\footnotesize{Choix: méthode des éléments finis}}}
        \item \only<3-5>{\textbf{La discrétisation en maillage} \textcolor{darkgreen}{\footnotesize(maillage triangulaire, quadrilatère, tétraèdrique, hexaédrique, hybride)}}\only<6->{\textbf{La discrétisation en maillage:} \textcolor{darkred}{\footnotesize{Choix: maillage quadrilatère / hexaédrique}}}
        \item \only<4-5>{\textbf{Les critères de convergence} \textcolor{darkgreen}{\footnotesize(critère de déplacement, critère de minimisation)}}\only<6->{\textbf{Les critères de convergence:} \textcolor{darkred}{\footnotesize{Choix: critère de déplacement et d'énergie sur une méthode de Newton}}}
        \item \only<5-5>{\textbf{Validation et vérification} \textcolor{darkgreen}{\footnotesize(comparaison des résultats avec des données de référence)}}\only<6->{\textbf{Validation et vérification:} \textcolor{darkred}{\footnotesize{Choix: si une simulation atteint l'itération finale, on la considère réussie}}}
    \end{itemize}

\end{frame}


\begin{frame}{Définition et propriétés fondamentales du maillage}
    \small{
        \textbf{Définition :} Un maillage $\mathcal{M}$ décompose un domaine géométrique fermé $\Omega$ en un ensemble fini d'éléments simples $(\sigma_i)_N$. 
        \begin{align*}
            \Omega = \bigcup_{0 \leq i \leq N}{\sigma_i}
        \end{align*}
    }
    \newline
    \small{
        \textbf{Conformité :} Un maillage est conforme si l'intersection entre deux éléments adjacents est vide, un sommet commun, une arête commune ou une face commune.\\
    }
    \vspace{0.5cm}
    \small{
        \textbf{Description CAO :} Représentation numérique d'un objet dans un espace 2D ou 3D utilisant des entités géométriques primitives.
    }
\end{frame}

\begin{frame}
    \frametitle{Mesh Elements}
    
    \begin{table}[]
    \centering
    \begin{tabular}{|c|c|c|}
    \hline
    & \textbf{Input} & \textbf{Output} \\ \hline
    \textbf{2D} & \includegraphics[width=0.2\textwidth]{img/new_images/triangle.png} & \includegraphics[width=0.2\textwidth]{img/new_images/quadrilatere.png} \\ \hline
    \textbf{3D} & \includegraphics[width=0.2\textwidth]{img/new_images/tetrahedron.png} & \includegraphics[width=0.2\textwidth]{img/new_images/hexahedron.png} \\ \hline
    \end{tabular}
    \end{table}
    
\end{frame}
\fi

\begin{frame}
    \frametitle{Mesh Elements}
    
    \begin{table}[]
    \centering
    \begin{tabular}{|M{1cm}|M{3cm}|M{3cm}|}
    \hline
    & \textbf{Input} & \textbf{Output} \\ [4ex] \hline
    \textbf{2D} & \raisebox{-0.5\height}{\includegraphics[width=\linewidth]{img/new_images/triangle.png}} & \raisebox{-0.5\height}{\includegraphics[width=\linewidth]{img/new_images/quadrilatere.png}} \\ [4ex] \hline
    \textbf{3D} & \raisebox{-0.5\height}{\includegraphics[width=\linewidth]{img/new_images/tetrahedron.png}} & \raisebox{-0.5\height}{\includegraphics[width=\linewidth]{img/new_images/hexahedron.png}} \\ [4ex] \hline
    \end{tabular}
    \end{table}
    
\end{frame}

\begin{frame}{Un maillage conforme unique pour toute la simulation}
    \begin{columns}
    \column{0.5\textwidth}
    \small{
        \textbf{Objectif :} Un maillage unique pour toute la simulation, conservant une bonne qualité malgré les déformations.\\
    }
    \vspace{0.3cm}
    \small{
        \textbf{Avantage :} Les hexaèdres peuvent être déformés sans affecter leurs angles dièdres, contrairement aux tétraèdres.\\
    }
    \vspace{0.3cm}
    \small{
        \textbf{Défi :} Créer ces maillages complexes peut prendre plusieurs mois de travail ingénieur.\\
    }
    \column{0.5\textwidth}
        \includegraphics[width=\textwidth]{img/choix_maillage/deformation_low_angle.png}
    \end{columns}
\end{frame}


\begin{frame}{Maillage hexaédrique de qualité pour ce type de simulation}
    \small{
        \textbf{Angles dièdres :} Entre 45 et 135 degrés pour éviter les problèmes de convergence.\\
    }
    \vspace{0.5cm}
    \small{
        \textbf{Rapport d'aspect :} Maximum de 100 entre la plus grande et la plus petite arête d'un hexaèdre.\\
    }
    \vspace{0.5cm}
    \small{
        \textbf{Alignement avec les bords :} Permet de préserver la qualité des angles pendant la simulation.\\
    }
    \vspace{0.5cm}
    \small{
        \textbf{Haute qualité localisée :} Crucial dans les zones à haut gradient de force pour minimiser les erreurs numériques.\\
    }
\end{frame}

\subsection{Objectifs de la thèse}
\begin{frame}
    \frametitle{Plan de la présentation}
    \tableofcontents[currentsubsection, sectionstyle=show/shaded, subsectionstyle=show/shaded/hide]
\end{frame}
\begin{frame}{Maillages Quadrilatères et Hexaédriques : Challenges}
    \begin{columns}[T] % align columns
        \begin{column}{.5\textwidth}
            %Un maillage non-structuré pose des problèmes de convergence. \vspace*{.2cm}\\
            \textbf{Qualité} [\cite{knupp_remarks_2007}]
            \begin{itemize}
                \item Alignement avec les bords
                \item Angles proche de 90°
                \item Peu de singularités
            \end{itemize}
            
            \textbf{Pas d’algorithme générique}
            \begin{itemize}
                \item Intervention humaine
                \item Subdivision manuelle en blocs structurés
                \item Plusieurs semaines/mois pour construire un maillage
            \end{itemize}
        \end{column}%
        
        \begin{column}{.5\textwidth}
            \includegraphics[width=\linewidth]{img/new_images/qualite_maillage_important.PNG}
        \end{column}
    \end{columns}
\end{frame}

\begin{frame}{Objectif de la thèse}
    \centering
    \textbf{Automatiser la génération de maillage quad/hex de haute qualité pour les simulations de grandes déformations}.\\ \vspace{1em}
    \only<1>{ 
        \begin{tabular}{c|c}
            \textit{Entrée} : Maillage tri & \textit{Sortie} : Maillage quad \\Triangles [\cite{shewchuk_triangle_2005}] & $< 1$ min \\ \hline \\
            \includegraphics[width=0.45\linewidth]{img/new_images/entree_maillage_tri.png} &  \includegraphics[width=0.45\linewidth]{img/new_images/sortie_maillage_quad.png} \\
            
        \end{tabular}
    }
    \only<2>{
        \begin{tabular}{c|c}
            \textit{Entrée} : Maillage tétra  & \textit{Sortie} : Maillage hex \\ Tetgen [\cite{si_tetgen_2015}] &  $< 1$ min \\ \hline \\
            \includegraphics[width=0.4\linewidth]{img/new_images/entree_maillage_tet.png} &  \includegraphics[width=0.4\linewidth]{img/new_images/sortie_maillage_hex.png} \\

        \end{tabular}
    }
    \only<3>{
        \begin{figure}
            \centering
            \includegraphics[width=\linewidth]{img/new_images/echec_simu.PNG}
            \caption{Les positions de singularités optimales pour une géométrie initiale peuvent conduire à des quadrilatères de mauvaise qualité après une déformation.}
        \end{figure}
    }
\end{frame}

\subsection{Maillage de haute qualité par paramétrisation globale}
\begin{frame}
    \frametitle{Plan de la présentation}
    \tableofcontents[currentsubsection, sectionstyle=show/shaded, subsectionstyle=show/shaded/hide]
\end{frame}
\begin{frame}{Lien entre champ de repère 2D et maillage quadrilatère}
    \begin{center}
        \includegraphics[width=\linewidth]{img/quadsimu/singus.PNG}
        \small{
            \textit{Les singularités d'un champ de repère sont utilisées pour déterminer la position des sommets de valence différente de 4 dans le maillage quadrilatère.}
        }
    \end{center}
\end{frame}
\begin{frame}{Quadcover pour générer un maillage quadrilatère}
    \begin{center}
        \includegraphics[width=\linewidth]{img/cubecover/pipeline.PNG}
        \small{
            \textit{Les différentes étapes de la méthode Quadcover pour construire un maillage quadrilatère.}
        }
    \end{center}
\end{frame}
\begin{frame}{Les différents types de carte $U : \mathbb{R}^2 \mapsto \mathbb{R}^2$}
	\centering
	\begin{tikzpicture}[scale=1.38]
		\begin{scope}
			\draw[thick] (-.35, 0) -- (1.95, 0) -- (1.95, 1.55) -- (-.35, 1.55) -- cycle;
			\clip (-.35, 0) -- (1.95, 0) -- (1.95, 1.55) -- (-.35, 1.55) -- cycle;
			\fill[gray!14] (.06, .97) -- (.53, 1) -- (.5, 1.55) -- (-.1, 1.53);
			\fill[gray!14] (.06, .97) -- (-.5, .9) -- (-.5, .3) -- (.1, .37);
			\fill[gray!14] (.1, .37) -- (.57, .4) -- (.58, 0) -- (.1, 0);
			\fill[gray!14] (.57, .4) -- (.9, .4) -- (.9, 1.) -- (.53, 1.);
			\fill[gray!14] (.9, 1.2) -- (.9, 1.55) -- (1.3, 1.55) -- (1.44, 1.24) -- (.98, .96);
			\fill[gray!14] (1.44, 1.24) -- (1.95, 1.6) -- (1.95, .88) -- (1.68, .67);
			\fill[gray!14] (.9, .9) -- (.99, .98) -- (1.2, .42) -- (.9, .22);
			\fill[gray!14] (1.53, -.05) -- (1.2, .42) -- (1.68, .67) -- (1.9, .22);
			\draw[ultra thick, red] (0.1, 0) arc(1:11:9);
			\draw[ultra thick, red] (.6, 0) arc(0:10:9);
			\draw[ultra thick, blue] (.9, 0.4) arc(90:99:8);
			\draw[ultra thick, blue] (.9, 1) arc(90:99:8);
			\draw[ultra thick, blue] (.9, 1.6) arc(90:99:8);
			\draw[ultra thick, red] (.9, .2) arc(-60:-50:8);
			\draw[ultra thick, red] (.9, .9) arc(-60:-51:8);
			\draw[ultra thick, red] (1.6, 0) arc(-55:-51:8);
			\draw[ultra thick, blue] (.9, 1.2) arc(200:209:9);
			\draw[ultra thick, blue] (1.3, 1.55) arc(200:210:9);
			\draw[ultra thick, green] (.9, 0) -- (.9, 1.55);
			\draw[thick] (-.35, 0) -- (1.95, 0) -- (1.95, 1.55) -- (-.35, 1.55) -- cycle;
		\end{scope}
		\node at (.8, -0.5) {$U_i = R_{\theta} U_j + \lambda_{ij}$};
		\node at (.8, -1.) {$p_{ij} = arrondi(\theta/90)$};
		\node at (.8, 2.2) {Champ de repère};
		\begin{scope}[xshift=2.7cm]
			\draw[thick] (-.35, 0) -- (1.95, 0) -- (1.95, 1.55) -- (-.35, 1.55) -- cycle;
			\clip (-.35, 0) -- (1.95, 0) -- (1.95, 1.55) -- (-.35, 1.55) -- cycle;
			\fill[gray!14] (.06, .97) -- (.53, 1) -- (.5, 1.55) -- (-.1, 1.53);
			\fill[gray!14] (.06, .97) -- (-.5, .9) -- (-.5, .3) -- (.1, .37);
			\fill[gray!14] (.1, .37) -- (.57, .4) -- (.58, 0) -- (.1, 0);
			\fill[gray!14] (.57, .4) -- (.9, .4) -- (.9, 1.) -- (.53, 1.);
			\fill[gray!14] (.9, .2) -- (.9, .8) -- (1.27, .78) -- (1.29, .17);
			\fill[gray!14] (1.8, .16) -- (1.77, .76) -- (1.95, .75) -- (1.95, .12);
			\fill[gray!14] (1.27, .78) -- (1.77, .76) -- (1.74, 1.36) -- (1.22, 1.38);
			\fill[gray!14] (.9, 1.4) -- (1.22, 1.39) -- (1.2, 1.55) -- (.9, 1.55);
			\fill[gray!14] (1.73, 1.34) -- (1.71, 1.55) -- (1.95, 1.55) -- (1.95, 1.32);
			\fill[gray!14] (1.3, 0) -- (1.8, 0) -- (1.8, .14) -- (1.3, .18);
			\draw[ultra thick, red] (0.1, 0) arc(1:11:9);
			\draw[ultra thick, red] (.6, 0) arc(0:10:9);
			\draw[ultra thick, blue] (.9, 0.4) arc(90:99:8);
			\draw[ultra thick, blue] (.9, 1) arc(90:99:8);
			\draw[ultra thick, blue] (.9, 1.6) arc(90:99:8);
			\draw[ultra thick, red] (.9, .2) arc(90:82:8);
			\draw[ultra thick, red] (.9, .8) arc(90:82:8);
			\draw[ultra thick, red] (.9, 1.4) arc(90:82:8);
			\draw[ultra thick, blue] (1.3, 0) arc(0:10:9);
			\draw[ultra thick, blue] (1.8, 0) arc(-1:9:9);
			\draw[ultra thick, green] (.9, 0) -- (.9, 1.55);
			\draw[thick] (-.35, 0) -- (1.95, 0) -- (1.95, 1.55) -- (-.35, 1.55) -- cycle;
		\end{scope}
		\node at (3.5, -0.5) {$U_i = R_{90 \cdot p_{ij}} U_j + \lambda_{ij}$};
		\node at (3.5, -1.) {$k_{ij} = arrondi(\lambda_{ij})$};
		\node at (3.5, 2.2) {Intégration};
		\begin{scope}[xshift=5.4cm]
			\draw[thick] (-.35, 0) -- (1.95, 0) -- (1.95, 1.55) -- (-.35, 1.55) -- cycle;
			\clip (-.35, 0) -- (1.95, 0) -- (1.95, 1.55) -- (-.35, 1.55) -- cycle;
			\fill[gray!14] (.06, .97) -- (.53, 1) -- (.5, 1.55) -- (-.1, 1.53);
			\fill[gray!14] (.06, .97) -- (-.5, .9) -- (-.5, .3) -- (.1, .37);
			\fill[gray!14] (.1, .37) -- (.57, .4) -- (.58, 0) -- (.1, 0);
			\fill[gray!14] (.57, .4) -- (1.13, .4) -- (1.11, 1.) -- (.53, 1.);
			\fill[gray!14] (1.13, .4) -- (1.7, .36) -- (1.7, 0) -- (1.15, 0);
			\fill[gray!14] (1.11, 1.) -- (1.68, .96) -- (1.6, 1.55) -- (1.04, 1.55);
			\fill[gray!14] (1.68, .96) -- (1.95, .9) -- (1.95, .3) -- (1.7, .36);
			\draw[ultra thick, red] (0.1, 0) arc(1:11:9);
			\draw[ultra thick, red] (.6, 0) arc(0:10:9);
			\draw[ultra thick, blue] (.9, 0.4) arc(90:99:8);
			\draw[ultra thick, blue] (.9, 1) arc(90:99:8);
			\draw[ultra thick, blue] (.9, 1.6) arc(90:99:8);
			\draw[ultra thick, red] (.9, .4) arc(90:82:8);
			\draw[ultra thick, red] (.9, 1.) arc(90:82:8);
			\draw[ultra thick, red] (.9, 1.6) arc(90:80:8);
			\draw[ultra thick, blue] (1.15, 0) arc(0:10:9);
			\draw[ultra thick, blue] (1.7, 0) arc(-1:9:9);
			\draw[ultra thick, green] (.9, 0) -- (.9, 1.55);
			\draw[thick] (-.35, 0) -- (1.95, 0) -- (1.95, 1.55) -- (-.35, 1.55) -- cycle;
		\end{scope}
		\node at (6.2, -0.5) {$U_i = R_{90 \cdot p_{ij}} U_j + k_{ij}$};
		\node at (6.2, 2.2) {Quantification};
	
	\end{tikzpicture}
	
	%\caption{Classification des cartes en fonction de leur comportement à travers une discontinuité (en vert) : carte arbitraire (à gauche), carte sans couture (au milieu) et carte préservant la grille (à droite).}
	%\label{fig:all_diff_maps}
\end{frame}

\begin{frame}{En 3D: Cubecover pour générer un maillage hexaédrique}
    \begin{center}
        \includegraphics[width=\linewidth]{img/cubecover/B34_graphe_interieur.PNG}
        \small{
            \textit{Cubecover utilise le graphe de singularité d'un champ de repère 3D pour construire un maillage hexaédrique partageant le même graphe de singularité.}
        }
    \end{center}
\end{frame}

\section{Contribution 1 : Champ de repère pour générer un maillage CAO } 
\begin{frame}{Plan de la présentation}
    \tableofcontents[currentsection, sectionstyle=show/hide, subsectionstyle=hide/hide/hide]
    %\includegraphics[width=\linewidth]{img/cubecover/pipeline.PNG}
    \begin{tikzpicture}
        \node[anchor=south west,inner sep=0] (image) at (0,0) {\includegraphics[width=\linewidth]{img/cubecover/pipeline.PNG}};
        \begin{scope}[x={(image.south east)},y={(image.north west)}]
            \draw[red, thick] (0,0) rectangle (.25,1); % Ajuster les coordonnées si nécessaire
        \end{scope}
    \end{tikzpicture}
\end{frame}
\subsection{Champ de repère 2D non-orthogonal}
\begin{frame}{Plan de la présentation}
    \tableofcontents[currentsubsection, sectionstyle=show/shaded, subsectionstyle=show/shaded/hide]
\end{frame}
\iffalse
\begin{frame}{Champ de repère non-orthogonal}
    \centering
    \includegraphics[width=0.4\linewidth]{img_spm_ff/shuriken_big_quads.PNG}
    \includegraphics[width=0.4\linewidth]{img_spm_ff/shuriken_nonortho.png}

\end{frame}
\fi

\begin{frame}{Motivations : Maillage quadrilatère}
    \centering
    
    \begin{minipage}[c]{0.48\textwidth}
    \centering 
    \textbf{Orthogonal fields}\\
    \vspace{0.3cm}
    \begin{adjustbox}{width=0.8\linewidth,clip,trim=0 0 {.48\width} 0}
        \includegraphics{img_spm_ff/comp1.png}
    \end{adjustbox}
    \end{minipage}%
    \hfill\vline\hfill
    \begin{minipage}[c]{0.48\textwidth}
    \centering 
    \textbf{Non-orthogonal fields}\\
    \vspace{0.3cm}
    \begin{adjustbox}{width=0.75\linewidth,clip,trim={.52\width} 0 0 0}
        \includegraphics{img_spm_ff/comp1.png}
    \end{adjustbox}
    \end{minipage}
    
    \vspace*{0.3cm}
    \begin{itemize}
         \item Les champs orthogonaux ne gèrent pas les coins de petit angle
         \item Ils ne sont donc pas adaptés en l'état aux modèles CAO
    \end{itemize}
    
\end{frame}

\begin{frame}{Qu'est ce qui est difficile ?}
    \centering
    %To compare two 2D frames, we can't directly compare their vectors $(\vec{u}, \vec{v})$, as a same frame is given by many different combinations of vectors : 
    \small
    \begin{itemize}
     \item Un repère est défini comme un ensemble de vecteurs : $$\mathcal{U} = \left\{\vec{u},\ -\vec{u},\ \vec{v},\ -\vec{v}\right\}$$ 
     \item Nous voulons minimiser la "distance" entre les repères voisins d'un champ
     
     %\item Distance definition between 2 sets $\mathcal{U}_i$ and $\mathcal{U}_j$ ?
     %\item Using distance of vectors induces finding a matching between sets \\%If you want to compare sets by comparing their vectors, you need to match each vector\\
   \end{itemize}
      \vspace*{0.5\baselineskip}
      
    \begin{overprint}
    \onslide<1> \centering
    \includegraphics[width=0.6\linewidth]{img_spm_ff/dist_question.PNG}
      %\includegraphics[width=0.8\linewidth]{vector_dist_1.PNG}
      %\includegraphics[width=0.8\linewidth]{vector_dist_2.PNG}
    \onslide<2> \centering
      \includegraphics[width=\linewidth]{img_spm_ff/dist_sol.PNG}
    
    \end{overprint}
    
    \normalsize
\end{frame}


\begin{frame}{Repère 2D en tant que fonction polynomiale}
    \centering
    
$\mathcal{U} = \left\{\vec{u},\ -\vec{u},\ \vec{v},\ -\vec{v}\right\}$  est représenté par un polynôme \\
\textbf{restreint au cercle unité}
$$P_\mathcal{U}(\vec{s}) = \langle \vec{u},\ \vec{s}\rangle^{4} +  \langle \vec{v},\ \vec{s}\rangle^{4}, \ \ \   \forall \vec{s} \in {\rm I\!R}^2, \lVert \vec{s} \rVert = 1$$

     \includegraphics[width=0.3\linewidth]{img_spm_ff/anoted_orthogonal.PNG}
    \ \ \ 
       \includegraphics[width=0.3\linewidth]{img_spm_ff/anoted_polynome.PNG}
    \\
    
    \normalsize
    \begin{itemize}
    %\item $P_\mathcal{U}(\vec{s}) = \langle \vec{u},\ \vec{s}\rangle^{4} +  \langle \vec{v},\ \vec{s}\rangle^{4}, \ \ \   \forall \vec{s} \in {\rm I\!R}^2, \lVert \vec{s} \rVert = 1$
     \item $dist(\mathcal{U}_i, \mathcal{U}_j) = \int (P_{\mathcal{U}_i} - P_{\mathcal{U}_j})^2$
    \end{itemize}
\end{frame}

\begin{frame}{Décomposition dans la base de Fourier}
    \centering
    \begin{overprint}
    \onslide<2> \centering
    Avec $v = u^{\perp}$ nous obtenons la même définition de distance que les travaux antérieurs.
    
    \includegraphics[width=0.95\linewidth]{img_spm_ff/ortho_decomposition_with_circle.PNG}
    
    $$ dist(\mathcal{U}_i, \mathcal{U}_j) =  \int (P_{\mathcal{U}_i} - P_{\mathcal{U}_j})^2 =  \left({\color{green}c_3(\mathcal{U}_i)} - {\color{green}c_3(\mathcal{U}_j)} \right)^2 
    + \left({\color{green}c_4(\mathcal{U}_i)} - {\color{green}c_4(\mathcal{U}_j)} \right)^2$$
    \onslide<1> \centering
    
    Décomposer $P_\mathcal{U}$ dans la base de Fourier simplifie $dist(\mathcal{U}_i, \mathcal{U}_j)$. \\
    \includegraphics[width=0.95\linewidth]{img_spm_ff/polynome_decomposition_with_circle.PNG}
    $$ dist(\mathcal{U}_i, \mathcal{U}_j) =  \int (P_{\mathcal{U}_i} - P_{\mathcal{U}_j})^2 = \sum_{\ell=1}^4 \left({\color{green}c_\ell(\mathcal{U}_i)} - {\color{green}c_\ell(\mathcal{U}_j)} \right)^2$$
    \end{overprint}
    
\end{frame} 

\begin{frame}{Optimisation d'un champ non-orthogonal 2D}
    \centering
    \small
    %From the decomposition in the Fourier basis, we define
    %$$ dist(\mathcal{U}_i, \mathcal{U}_j) =  \int (P_{\mathcal{U}_i} - P_{\mathcal{U}_j})^2 = \sum_{\ell=1}^4 \left({\color{green}c_\ell(\mathcal{U}_i)} - {\color{green}c_\ell(\mathcal{U}_j)} \right)^2$$
    Pour calculer un champ de repère non-orthogonal 2D lisse, nous minimisons avec LBFGS l'énergie suivante :
    $$ E_{tot} = \sum_{Voisins(i, j)} dist(\mathcal{U}_i, \mathcal{U}_j)$$
    
    Contrôle de l'orthogonalité: $c_1, c_2, c_3, c_4 \mapsto \lambda c_1, \lambda c_2, c_3, c_4$ 
     
     \vspace*{0.5\baselineskip}
     %modifies the orthogonality of the field : 
    \begin{minipage}[b]{0.15\textwidth}
        \centering
        \includegraphics[width=\textwidth]{img_spm_ff/perced_1}
        $\lambda = 0.1$
    \end{minipage}
    \ \ \ 
    %\hfill
    \begin{minipage}[b]{0.15\textwidth}
        \centering
        \includegraphics[width=\textwidth]{img_spm_ff/perced_9}
        $\lambda = 0.5$
    \end{minipage}
    %\hfill
    \ \ \ 
    \begin{minipage}[b]{0.15\textwidth}
        \centering
        \includegraphics[width=\textwidth]{img_spm_ff/perced_16}
        $\lambda = 1$
    \end{minipage}
    %\hfill
    \ \ \ 
    \begin{minipage}[b]{0.15\textwidth}
        \centering
        \includegraphics[width=\textwidth]{img_spm_ff/perced_25}
        $\lambda = 1.5$
    \end{minipage}
\end{frame} 



\subsection{Champ de repère 3D non-orthogonal}
\begin{frame}
    \frametitle{Plan de la présentation}
    \tableofcontents[currentsubsection, sectionstyle=show/shaded, subsectionstyle=show/shaded/hide]
\end{frame}

\begin{frame}{Motivation : Maillage hexaédrique}
    \centering
    
    \begin{minipage}[c]{0.48\textwidth}
    \centering 
    \textbf{Champ orthogonal}\\
    \vspace{0.3cm}
    \includegraphics[width=.7\linewidth]{img_spm_ff/slope_ortho_front.png}
    \end{minipage}%
    \hfill\vline\hfill
    \begin{minipage}[c]{0.48\textwidth}
    \centering 
    \textbf{Champ non-orthogonal}\\
    \vspace{0.3cm}
    \includegraphics[width=.7\linewidth]{img_spm_ff/slope_northo_front.png}
    \end{minipage}
    
    \vspace*{0.3cm}
    \begin{itemize}
        \item Un champ 3D orthogonal ne gère pas les coins de petit angle.
        \item Ce travail permet l'optimisation de repères 3D non-orthogonaux.
   \end{itemize}
    
\end{frame}

\begin{frame}{Polynôme représentant un repère à 3 directions}
    \centering
    \small
    $\mathcal{U} = \left\{\vec{u},\ -\vec{u},\ \vec{v},\ -\vec{v},\ \vec{w},\ -\vec{w}\right\}$ est représenté par un polynôme \\
    \textbf{restreint à la sphère unité}
    %$$P_\mathcal{U} \colon s \mapsto \langle \vec{u},\ s\rangle^{4} +  \langle \vec{v},\ s\rangle^{4} +  \langle \vec{w},\ s\rangle^{4}$$
    $$ P_\mathcal{U}(\vec{s}) = \langle \vec{u},\ \vec{s}\rangle^{4} +  \langle \vec{v},\ \vec{s}\rangle^{4} +  \langle \vec{w},\ \vec{s}\rangle^{4}, \ \ \   \forall \vec{s} \in {\rm I\!R}^3, \lVert \vec{s} \rVert = 1$$
    
    \vspace*{.5\baselineskip}
    \includegraphics[width=0.33\linewidth]{img_spm_ff/sperical_3dir4.png}
    \ \ \ \ \ \ \ \ \ \ \ 
    \includegraphics[width=0.3\linewidth]{img_spm_ff/sperical_3dir4_rot.png}
    %\begin{align*}
     %  p_c \colon & s \mapsto \langle u,\ s\rangle^{4} +  \langle v,\ s\rangle^{4} +  \langle w,\ s\rangle^{4} \\
     %  & \mathcal{S}_{\{0, 1\}}^3 \to {\rm I\!R}
    %\end{align*}
\end{frame} 
\begin{frame}{Base des Harmoniques Sphériques}
    \centering
    \begin{overprint}
    \onslide<2> \centering
    Avec $u \perp v \perp w$, même représentation que [\cite{huang_boundary_2011}].\\
    
    %Polynomials of 3D orthogonal frames can be decomposed as:
    \begin{minipage}[c]{0.24\textwidth}
        \centering
          \vspace*{.5\baselineskip}
          \hfill
        \includegraphics[width=0.6\linewidth]{img_spm_ff/sperical_3dir4.png}
    \end{minipage}
    \begin{minipage}[c]{0.74\textwidth}
        %\centering
        %\includegraphics[width=0.33\linewidth]{sperical_3dir4.png}
        $$P_\mathcal{U}(s) = const \cdot Y_{0, 0} + \sum_{m = -4}^4{{\color{green}c_{4, m}(\mathcal{U})}Y_{4,m}(s)}$$ 
    \end{minipage}
    
    \includegraphics[width=.8\linewidth]{img_spm_ff/ortho_harmonic_decompo.PNG} 
    
    \onslide<1> \centering
        La décomposition de $P_\mathcal{U}$ dans la base des harmoniques sphériques simplifie $dist(\mathcal{U}_i, \mathcal{U}_j)$ \\
    
     %$2^{nd}$ order terms are necessary for non-orthogonality:
    %$$P_\mathcal{U}(s) = const \cdot Y_{0, 0} + \sum_{\ell \in \{2, 4\}} \sum_{m = -\ell}^\ell{{\color{green}c_{\ell, m}(\mathcal{U})}Y_{\ell,m}(s)}$$ 
    
    \begin{minipage}[c]{0.19\textwidth}
        \centering
          \vspace*{.5\baselineskip}
          \hfill
        \includegraphics[width=0.8\linewidth]{img_spm_ff/sperical_3dir4_rot.png}
    \end{minipage}
    \begin{minipage}[c]{0.79\textwidth}
        %\centering
        %\includegraphics[width=0.33\linewidth]{sperical_3dir4.png}
        %$$P_\mathcal{U}(s) = const \cdot Y_{0, 0} + \sum_{m = -4}^4{{\color{green}c_{4, m}(\mathcal{U})}Y_{4,m}(s)}$$ 
        $$P_\mathcal{U}(s) = const \cdot Y_{0, 0} + \sum_{\ell \in \{2, 4\}} \sum_{m = -\ell}^\ell{{\color{green}c_{\ell, m}(\mathcal{U})}Y_{\ell,m}(s)}$$ 
    \end{minipage}
    
    \vspace*{.5\baselineskip}
    \includegraphics[width=\linewidth]{img_spm_ff/all_sph_harm.PNG} 
    \end{overprint}
\end{frame} 

\begin{frame}{Optimisation d'un champ de repère non-orthogonal 3D}
    \centering
    \footnotesize
    Distance par décomposition dans la base des Harmoniques Sphériques:
    $$ dist(\mathcal{U}_i, \mathcal{U}_j) = \int (P_{\mathcal{U}_i} - P_{\mathcal{U}_j})^2 = \sum_{\ell, m} \left({\color{green}c_{\ell,m}(\mathcal{U}_i)} - {\color{green}c_{\ell,m}(\mathcal{U}_j)} \right)^2$$
    Pour calculer un champ de repère non-orthogonal 3D lisse, minimiser :
    %To compute smoothed frame field, we minimize with LBFGS 
    $$ E_{tot} = \sum_{Voisins(i, j)} dist(\mathcal{U}_i, \mathcal{U}_j)$$
    %\vspace*{.5\baselineskip}
    %As in 2D, $\forall m,\ \ c_{2,m} \mapsto \lambda c_{2,m}$ modifies the orthogonality of the field. 
    Contrôle de l'orthogonalité: $\ \ c_{2,m} \mapsto \lambda c_{2,m} \ \ \ \ \ (-2 \leq m \leq 2)$ %modifies the orthogonality of the field. 
    
    \begin{minipage}[b]{0.33\textwidth}
        \centering
        \includegraphics[width=\textwidth]{img_spm_ff/shear_0_7.png}
        $\lambda = 0.7$
    \end{minipage}
    \begin{minipage}[b]{0.28\textwidth}
        \centering
        \includegraphics[width=\textwidth]{img_spm_ff/shear_1.png}
        $\lambda = 1$
    \end{minipage}
    
    \normalsize
\end{frame} 

\iffalse
\begin{frame}{Maillages obtenus avec des champs non-orthogonaux}
    \includegraphics[width=.49\textwidth]{img_spm_ff/parallel.png}
    \includegraphics[width=.49\textwidth]{img_spm_ff/joint_northo.png}
\end{frame}
\fi
\begin{frame}{Maillages obtenus avec des champs non-orthogonaux}
\begin{minipage}[c]{0.48\textwidth}
    \centering 
    \textbf{Champ orthogonal}\\
    \vspace{0.3cm}
    \includegraphics[width=.7\textwidth]{img_spm_ff/joint_ortho.png}
    \includegraphics[width=.7\textwidth]{img_spm_ff/shear_ortho.png}
    \end{minipage}%
    \hfill\vline\hfill
    \begin{minipage}[c]{0.48\textwidth}
    \centering 
    \textbf{Champ non-orthogonal}\\
    \vspace{0.3cm}
    \includegraphics[width=.7\textwidth]{img_spm_ff/joint_northo.png}
    \includegraphics[width=.7\textwidth]{img_spm_ff/shear_1.png}
\end{minipage}
\end{frame} 

%\subsection{Champ 2D pour maillage quadrilatère de modèle CAO}
%\begin{frame}
%    \frametitle{Plan de la présentation}
%    \tableofcontents[currentsubsection, sectionstyle=show/shaded, subsectionstyle=show/shaded/hide]
%\end{frame}
%

\begin{frame}{Repère orthogonal 2D}
    \small
    Un repère orthogonal 2D est une croix orientée par un angle $\theta$.
    \newline
    \newline
    \begin{minipage}{0.46\textwidth}
        \centering
        \begin{tikzpicture}[very thick, scale=.9]
            \draw[->, red] (.8, 0) arc (0:40:.8) node[right,pos=.66] {$\theta_1$};
            \draw[->] (180:1.2) -- (0:1.2);
            \draw[->] (-90:1.2) -- (90:1.2);
            \draw[green] (220:1.2) -- (40:1.2);
            \draw[green] (310:1.2) -- (130:1.2);
        \end{tikzpicture}
    \end{minipage}
    \hfill
    \begin{minipage}{0.46\textwidth}
        \centering
        \begin{tikzpicture}[very thick, scale=.9]
            \draw[->, red] (.8, 0) arc (0:130:.8) node[right,pos=.2] {$\theta_2$};
            \draw[->] (180:1.2) -- (0:1.2);
            \draw[->] (-90:1.2) -- (90:1.2);
            \draw[green] (220:1.2) -- (40:1.2);
            \draw[green] (310:1.2) -- (130:1.2);
        \end{tikzpicture}
    \end{minipage}
    
    \vfill
    
    \small
    Une croix étant $\pi/2$-périodique, nous la représentons par $(X, Y) = (\cos4\theta, \sin4\theta)$ pour éviter les problèmes de périodicité.
    \newline
    \newline
    \begin{minipage}{0.4\textwidth}
        \centering
        \begin{tikzpicture}[very thick, scale=.9]
            \draw[->, red] (.8, 0) arc (0:160:.8) node[right,pos=.3] {$4\theta_1$};
            \draw[->] (180:1.2) -- (0:1.2);
            \draw[->] (-90:1.2) -- (90:1.2);
            \draw[->, blue] (340:0) -- (160:1.2) node[right,pos=1.4] {$(X_1, Y_1)$};
        \end{tikzpicture}
    \end{minipage}
    \hfill
    \begin{minipage}{0.53\textwidth}
        \centering
        \begin{tikzpicture}[very thick, scale=.9]
            \draw[->, red] (.8, 0) arc (0:520:.8) node[right,pos=.1] {$4\theta_2$};
            \draw[->] (180:1.2) -- (0:1.2);
            \draw[->] (-90:1.2) -- (90:1.2);
            \draw[->, blue] (340:0) -- (160:1.2) node[right,pos=1.4] {$(X_2, Y_2)$};
        \end{tikzpicture}
    \end{minipage}
\end{frame}

\begin{frame}{Champ de repère 2D plat}
    \small
    Pour optimiser un champ de repère 2D plat sans problème de périodicité, nous optimisons 
    les vecteurs de représentations $(X, Y)$:
    \small{
    \begin{equation*}
    \begin{array}{ll}
    \underset{X, Y}{\argmin} & \underset{t \in T}{\displaystyle\sum} \underset{t' \in \mathcal{N}(t)}{\displaystyle\sum} \left|\left|\ \begin{pmatrix} X_{t'}\\ Y_{t'}\end{pmatrix} - \begin{pmatrix} X_{t}\\ Y_{t}\end{pmatrix} \right|\right|^2, \\
    \text{sous contrainte: } & \forall t \in T_b, \begin{pmatrix} X_{t}\\ Y_{t}\end{pmatrix} = \begin{pmatrix} \cos4\eta_t\\ \sin4\eta_t\end{pmatrix}.
    \end{array}
    \end{equation*}
    }
    Puis nous retrouvons les angles $\theta$ du champ de repère:
    $$ \forall t \in T,\ \  \theta_t = \frac{1}{4}\tan^{-1}\frac{Y_t}{X_t}$$

\end{frame}


\begin{frame}{Problèmes des bords à petits angles}

    \begin{center}
    \begin{tabular}{c|c|c|c|c}
    % First row: angle text
    $\alpha_v=10^{\circ}$ & $\alpha_v=90^{\circ}$ & $\alpha_v=180^{\circ}$ & $\alpha_v=270^{\circ}$ & $\alpha_v=350^{\circ}$ \\
    \hline
    % Second row: TikZ diagrams
    \begin{minipage}{0.14\textwidth}
    \centering
    \begin{tikzpicture}[scale=0.2]
    % Lines forming a 10° angle
    \draw (0,0) -- (0,3);
    \draw (0,0) -- ({3*sin(10)},{3*cos(10)});
    \node[rotate=40, green, scale=2] at (0.2,2) {$\times$}; % Added cross
    \end{tikzpicture}
    \end{minipage}
    &
    \begin{minipage}{0.14\textwidth}
    \centering
    \begin{tikzpicture}[scale=0.2]
    \only<2-2> {
    \draw[red] (0.0,0.0) rectangle (3.0,3.0); % Added quadrilateral
    }
    % Lines forming a 90° angle
    \draw (0,0) -- (0,3);
    \draw (0,0) -- (3,0);
    \node[rotate=45, green, scale=2] at (1.5,1.5) {$\times$}; % Added cross
    \end{tikzpicture}
    \end{minipage}
    &
    \begin{minipage}{0.17\textwidth}
    \centering
    \begin{tikzpicture}[scale=0.2]
    \only<2-2> {
    \draw[red] (0.0,0.0) rectangle (3.0,3.0); % Added quadrilateral
     \draw[red] (0.0,-3.0) rectangle (3.0,0.0); % Added quadrilateral
    }
    % Lines forming a 180° angle
    \draw (0,0) -- (0,3);
    \draw (0,0) -- (0,-3);
    \node[rotate=45, green, scale=2] at (1.5,1.5) {$\times$}; % Added cross
    \node[rotate=45, green, scale=2] at (1.5,-1.5) {$\times$}; % Added cross
    \end{tikzpicture}
    \end{minipage}
    &
    \begin{minipage}{0.17\textwidth}
    \centering
    \begin{tikzpicture}[scale=0.2]
    % Lines forming a 270° angle
    \only<2-2> {
     \draw[red] (0.0,0.0) rectangle (3.0,3.0); % Added quadrilateral
     \draw[red] (0.0,-3.0) rectangle (3.0,0.0); % Added quadrilateral
     \draw[red] (-3.0,-3.0) rectangle (0.0,0.0); % Added quadrilateral
    }
    \draw (0,0) -- (0,3);
    \draw (0,0) -- (-3,0);
    \node[rotate=45, green, scale=2] at (1.5,1.5) {$\times$}; % Added cross
    \node[rotate=45, green, scale=2] at (1.5,-1.5) {$\times$}; % Added cross
    \node[rotate=45, green, scale=2] at (-1.5,-1.5) {$\times$}; % Added cross
    \end{tikzpicture}
    \end{minipage}
    &
    \begin{minipage}{0.17\textwidth}
    \centering
    \begin{tikzpicture}[scale=0.2]
    % Lines forming a 350° angle
    \only<2-2> {
     \draw[red] (0.0,0.0) rectangle (3.0,3.0); % Added quadrilateral
     \draw[red] (0.0,-3.0) rectangle (3.0,0.0); % Added quadrilateral
     \draw[red] (-3.0,-3.0) rectangle (0.0,0.0); % Added quadrilateral
     \draw[red] (-3.0,0.0) -- (0.0,0.0) -- ({3*sin(350)},{3*cos(350)}) -- ({3*sin(350)-3},{3*cos(350) - 0.3}) -- cycle; % Added parallelogram
    }
    \draw (0,0) -- (0,3);
    \draw (0,0) -- ({3*sin(350)},{3*cos(350)});
    \node[rotate=45, green, scale=2] at (1.5,1.5) {$\times$}; % Added cross
    \node[rotate=45, green, scale=2] at (1.5,-1.5) {$\times$}; % Added cross
    \node[rotate=50, green, scale=2] at (-1.5,-1.5) {$\times$}; % Added cross
    \node[rotate=54, green, scale=2] at (-1.7,1.5) {$\times$}; % Added cross
    \end{tikzpicture}
    \end{minipage}
    \end{tabular}
    
    \only<2-2>{
        \vspace{0.3cm}
        Un champ de repère orthogonal produit un maillage quadrilatère de valence de bord: $$k_v = \text{arrondi} \left( \frac{\alpha_v}{90} \right)$$
    
        Problème: pour $\alpha_v < 45^\circ$, on obtient une valence $k_v = 0$ qui fait échouer la méthode. %le sommet de bord est donné de valence 0, ce qui fait échouer la méthode.
    }
    \end{center}
    
\end{frame}

\begin{frame}{Problèmes avec les modèles CAO}
    \begin{center}
        \includegraphics[width=0.99\linewidth]{img/cadff/teaser2}
    \end{center}
\end{frame}
    
\begin{frame}{Intuition de la contribution}
    \begin{center}
        \includegraphics[width=0.9\linewidth]{img/new_images/flat_tri_annot.png}
        \includegraphics[width=0.9\linewidth]{img/new_images/surface_tri_annot.png}
        \small{
            \textit{Modifier la définition du transport parallèle permet de transformer les petits angles en angles de 90°.}
        }
    \end{center}
\end{frame}
\begin{frame}{Modification de la définition du transport parallèle pour empêcher les petits angles}

    \begin{center}
    \begin{tikzpicture}[scale=1]
    % Triangle 1
    \fill[blue!20] (0,0) -- (8,0) -- (8,1) -- cycle;
    
    % Triangle 2
    \fill[red!20] (0,0) -- (8,0) -- (8,-1) -- cycle;
    
    \node at ($ (2,0) $) {$\alpha_v = 14^{\circ}$}; % Shifted to the right
    \draw[->, blue] ($ (6,0) + (0,-0.2) $) to [bend right=45] ($ (6,0) + (0,0.2) $); % Flèche going from bottom to top
    \node[blue] at ($ (6,0) + (1,0) $) {$\gamma = 0^{\circ}$};

    \end{tikzpicture}

    \vspace{.5cm} % Adjust this value to increase or decrease the space

    \begin{tikzpicture}[scale=1]
    % Triangle 1
    \fill[blue!20] (0,0) -- (8,0) -- (8,1) -- cycle;
    
    % Triangle 2
    \fill[red!20] (0,0) -- (8,0) -- (8,-1) -- cycle;
    
    \node at ($ (2,0) $) {$\alpha_v + K_v = 90^{\circ}$}; % Shifted to the right
    \draw[->, blue] ($ (6,0) + (0,-0.2) $) to [bend right=45] ($ (6,0) + (0,0.2) $); % Flèche going from bottom to top
    \node[blue] at ($ (6,0) + (1,0) $) {$\gamma = 76^{\circ}$};

    \end{tikzpicture}
    \end{center}
    
\end{frame}

\begin{frame}{Diffusion of $\gamma$}

    \begin{minipage}{0.59\textwidth}
    \begin{figure}
      \centering
      \only<1>{\includegraphics[width=0.79\linewidth]{img/cadff/sharp0}
      \caption{Champ le plus lisse}}
      \includegraphics[width=0.79\linewidth]{img/cadff/sharp1}
      \caption{Singularité sur le 1-voisinage}
      \only<2>{\includegraphics[width=0.79\linewidth]{img/cadff/sharp2}
      \caption{Propagation globale de la courbure}}
    \end{figure}
    \end{minipage}%
    \begin{minipage}{0.39\textwidth}
        Pour tous les sommets de coin à petit angle \(v\):\\
        On fixe \(K_v = 90 - \theta_v\).
        \only<2>{Puis: \(\argmin \displaystyle\sum_{tt'}|\gamma_{tt'}|^2\)}
        
        \vspace{1em}
        \begin{block}{Contraintes:}
            \begin{itemize}
                \item \(\displaystyle\sum_{v \in \mathcal{V}} K_v =0\) 
                \item $ \forall v,\displaystyle\sum_{tt' \in \mathcal{N}(v)}\gamma_{tt'} = K_v$
            \end{itemize}
        \end{block}
    \end{minipage}
    
\end{frame}

\begin{frame}{Optimisation d'un champ adapté aux modèles CAO}

    Pour chaque paire de triangles adjacents $(t, t')$, on calcule la matrice de rotation 
    induite par $\gamma_{tt'}$:
    $$R_{tt'} = \begin{pmatrix}\cos4\gamma_{tt'} & -\sin4\gamma_{tt'} \\ \sin4\gamma_{tt'} & \cos4\gamma_{tt'} \end{pmatrix}$$
    Le problème d'optimisation devient alors :
   
    \begin{equation*}
        \begin{array}{ll}
        \underset{X, Y}{\argmin} & \underset{t \in T}{\displaystyle\sum} \underset{t' \in \mathcal{N}(t)}{\displaystyle\sum} \left|\left|\ \begin{pmatrix} X_{t'}\\ Y_{t'}\end{pmatrix} - R_{tt'} \begin{pmatrix} X_{t}\\ Y_{t}\end{pmatrix} \right|\right|^2, \\
        \text{sous contrainte: } & \forall t \in T_b, \begin{pmatrix} X_{t}\\ Y_{t}\end{pmatrix} = \begin{pmatrix} \cos4\eta_t\\ \sin4\eta_t\end{pmatrix}.
        \end{array}
        \label{eq:cadff_ff_2D_surface_optim}
    \end{equation*}
    
\end{frame}


%\subsection{Champ 3D pour maillage hexaédrique de modèle CAO}
%\begin{frame}
%    \frametitle{Plan de la présentation}
%    \tableofcontents[currentsubsection, sectionstyle=show/shaded, subsectionstyle=show/shaded/hide]
%\end{frame}
%\begin{frame}{Gestion des petits angles dans le cas 3D}

    \begin{figure}
        \centering
        \includegraphics[width=0.8\textwidth]{img/hexmeshing_ff/tremplin_solved_2.PNG}
    \end{figure}
    
    Les champs de repère orthogonaux échouent lorsque l'angle de la pente est trop faible, donnant des résultats dégénérés (à gauche). En utilisant notre méthode, nous obtenons des contraintes de direction qui séparent correctement la pente et son support (à droite).
    
\end{frame}

\begin{frame}{Intuition de la méthode}

    \begin{figure}
        \centering
        \includegraphics[width=0.8\textwidth]{img/hexmeshing_ff/normal_alignment_with_hexes_2.PNG}
    \end{figure}
    
    Un champ de repère orthogonal aligné avec les normales de surface peut créer une arête de valence 0 dans le cas d'une arête à petit angle (en haut à gauche), qui ne peut pas être maillée. En recalculant les contraintes de direction, nous obtenons une arête de valence 1 qui peut être maillée (en haut à droite).
    
\end{frame}

\begin{frame}{Calcul des angles dièdres et des valences géométriques}

    \begin{figure}
        \centering
        \includegraphics[width=0.8\textwidth]{img/hexmeshing_ff/phi_angles.PNG}
    \end{figure}
    
    \only<1-1>{
    L'angle dièdre $\varphi_{tt'}$ et la valence géométrique d'arête $k_{tt'}$ sont déterminés en utilisant les normales de faces $n_t$ et $n_{t'}$ adjacentes à l'arête de bord.
    }
    \only<2-2>{
        \begin{align*}
            \varphi(n_t, n_{t'}) &= \pi - \mathrm{atan2} \left( \langle n_t \times n_{t'}, \frac{e_{tt'}}{\left|e_{tt'}\right|} \rangle, \langle n_t , n_{t'} \rangle \right)\\
            k_{tt'} &= \text{arrondi}( \frac{\varphi_{tt'}}{\pi/2} )
        \end{align*}
    }
\end{frame}

\begin{frame}{Valences prescrites dans des modèles CAO}

    \begin{figure}
        \centering
        \includegraphics[width=0.7\textwidth]{img/hexmeshing_ff/prescribed_valences.PNG}
    \end{figure}
    
    \small
    Dans le cas de non-orthogonalité extrême dans des modèles CAO, il est difficile de déterminer géométriquement les valences des arêtes caractéristiques. Pour obtenir ces résultats, nous avons prescrit les valences des arêtes caractéristiques dans les descriptions CAO des modèles en entrée.
    
\end{frame}

\section{Contribution 2 : Quantification de paramétrisation 2D}
\begin{frame}{Plan de la présentation}
    \tableofcontents[currentsection, sectionstyle=show/hide, subsectionstyle=hide/hide/hide]
    %\includegraphics[width=\linewidth]{img/cubecover/pipeline.PNG}
    \begin{tikzpicture}
        \node[anchor=south west,inner sep=0] (image) at (0,0) {\includegraphics[width=\linewidth]{img/cubecover/pipeline.PNG}};
        \begin{scope}[x={(image.south east)},y={(image.north west)}]
            \draw[red, thick] (0.76,0) rectangle (1,1); % Ajuster les coordonnées si nécessaire
        \end{scope}
    \end{tikzpicture}
\end{frame}
\subsection{État de l'art : QGP [\cite{campen_quantized_2015}]}
\begin{frame}
    \frametitle{Plan de la présentation}
    \tableofcontents[currentsubsection, sectionstyle=show/shaded, subsectionstyle=show/shaded/hide]
\end{frame}
\input{tex/qgp}
\subsection{Contribution : Quantification sans recalcul de paramétrisation}
\begin{frame}
    \frametitle{Plan de la présentation}
    \tableofcontents[currentsubsection, sectionstyle=show/shaded, subsectionstyle=show/shaded/hide]
\end{frame}
\begin{frame}{Quantification: Rappel}
    \centering
    \begin{tikzpicture}
        \begin{scope}[xshift=2.7cm]
            \draw[thick] (-.35, 0) -- (1.95, 0) -- (1.95, 1.55) -- (-.35, 1.55) -- cycle;
            \clip (-.35, 0) -- (1.95, 0) -- (1.95, 1.55) -- (-.35, 1.55) -- cycle;
            \fill[gray!14] (.06, .97) -- (.53, 1) -- (.5, 1.55) -- (-.1, 1.53);
            \fill[gray!14] (.06, .97) -- (-.5, .9) -- (-.5, .3) -- (.1, .37);
            \fill[gray!14] (.1, .37) -- (.57, .4) -- (.58, 0) -- (.1, 0);
            \fill[gray!14] (.57, .4) -- (.9, .4) -- (.9, 1.) -- (.53, 1.);
            \fill[gray!14] (.9, .2) -- (.9, .8) -- (1.27, .78) -- (1.29, .17);
            \fill[gray!14] (1.8, .16) -- (1.77, .76) -- (1.95, .75) -- (1.95, .12);
            \fill[gray!14] (1.27, .78) -- (1.77, .76) -- (1.74, 1.36) -- (1.22, 1.38);
            \fill[gray!14] (.9, 1.4) -- (1.22, 1.39) -- (1.2, 1.55) -- (.9, 1.55);
            \fill[gray!14] (1.73, 1.34) -- (1.71, 1.55) -- (1.95, 1.55) -- (1.95, 1.32);
            \fill[gray!14] (1.3, 0) -- (1.8, 0) -- (1.8, .14) -- (1.3, .18);
            \draw[ultra thick, red] (0.1, 0) arc(1:11:9);
            \draw[ultra thick, red] (.6, 0) arc(0:10:9);
            \draw[ultra thick, blue] (.9, 0.4) arc(90:99:8);
            \draw[ultra thick, blue] (.9, 1) arc(90:99:8);
            \draw[ultra thick, blue] (.9, 1.6) arc(90:99:8);
            \draw[ultra thick, red] (.9, .2) arc(90:82:8);
            \draw[ultra thick, red] (.9, .8) arc(90:82:8);
            \draw[ultra thick, red] (.9, 1.4) arc(90:82:8);
            \draw[ultra thick, blue] (1.3, 0) arc(0:10:9);
            \draw[ultra thick, blue] (1.8, 0) arc(-1:9:9);
            \draw[ultra thick, green] (.9, 0) -- (.9, 1.55);
            \draw[thick] (-.35, 0) -- (1.95, 0) -- (1.95, 1.55) -- (-.35, 1.55) -- cycle;
        \end{scope}
        \node at (3.5, 1.8) {Intégration};
        \node at (3.5, -0.5) {\green{$\alpha = arrondi(\theta)$}};
        \node at (3.5, -.9) {$f_i = R_{\green{\alpha}} f_j + \lambda$};
        
        \begin{scope}[xshift=5.4cm]
            \draw[thick] (-.35, 0) -- (1.95, 0) -- (1.95, 1.55) -- (-.35, 1.55) -- cycle;
            \clip (-.35, 0) -- (1.95, 0) -- (1.95, 1.55) -- (-.35, 1.55) -- cycle;
            \fill[gray!14] (.06, .97) -- (.53, 1) -- (.5, 1.55) -- (-.1, 1.53);
            \fill[gray!14] (.06, .97) -- (-.5, .9) -- (-.5, .3) -- (.1, .37);
            \fill[gray!14] (.1, .37) -- (.57, .4) -- (.58, 0) -- (.1, 0);
            \fill[gray!14] (.57, .4) -- (1.13, .4) -- (1.11, 1.) -- (.53, 1.);
            \fill[gray!14] (1.13, .4) -- (1.7, .36) -- (1.7, 0) -- (1.15, 0);
            \fill[gray!14] (1.11, 1.) -- (1.68, .96) -- (1.6, 1.55) -- (1.04, 1.55);
            \fill[gray!14] (1.68, .96) -- (1.95, .9) -- (1.95, .3) -- (1.7, .36);
            \draw[ultra thick, red] (0.1, 0) arc(1:11:9);
            \draw[ultra thick, red] (.6, 0) arc(0:10:9);
            \draw[ultra thick, blue] (.9, 0.4) arc(90:99:8);
            \draw[ultra thick, blue] (.9, 1) arc(90:99:8);
            \draw[ultra thick, blue] (.9, 1.6) arc(90:99:8);
            \draw[ultra thick, red] (.9, .4) arc(90:82:8);
            \draw[ultra thick, red] (.9, 1.) arc(90:82:8);
            \draw[ultra thick, red] (.9, 1.6) arc(90:80:8);
            \draw[ultra thick, blue] (1.15, 0) arc(0:10:9);
            \draw[ultra thick, blue] (1.7, 0) arc(-1:9:9);
            \draw[ultra thick, green] (.9, 0) -- (.9, 1.55);
            \draw[thick] (-.35, 0) -- (1.95, 0) -- (1.95, 1.55) -- (-.35, 1.55) -- cycle;
        \end{scope}
        \node at (6.2, 1.8) {Quantification};
        \node at (6.2, -.5) {\purple{$k = arrondi(\lambda)$}};
        \node at (6.2, -0.9) {$f_i = R_{\green{\alpha}} f_j + \purple{k}$};
    \end{tikzpicture}
\end{frame}

\begin{frame}{Travail inspiré par: Quantized Global Parametrization}
    \centering
    \includegraphics[width=\linewidth]{yoimg/qgp.png}
\end{frame}

\begin{frame}{QGP, Campen et al. 2015}
    \centering
    \includegraphics[width=0.48\linewidth]{yoimg/tmesh.png}
    \includegraphics[width=0.42\linewidth]{yoimg/tmesh2.png} \\
    \includegraphics[width=0.33\linewidth]{yoimg/tron.png}
    \includegraphics[width=0.33\linewidth]{yoimg/tron2.png}
\end{frame}

\begin{frame}{Contribution: Quantification sans maillage en T}
    \centering
    \small Y. Coudert-\,-Osmont$^1$, D. Desobry$^1$, M. Heistermann$^2$, D. Bommes$^2$, N.Ray$^1$, D. Sokolov$^1$ \\
    \tiny $^1$Inria Nancy - Grand Est, LORIA, France \\
    \tiny $^2$University of Bern, Switzerland \\[2mm]
    \includegraphics[width=.9\linewidth]{yoimg/teaser.PNG}
\end{frame}

\begin{frame}{Contribution: Quantification sans maillage en T}
    \begin{itemize}
        \item QGP peut écraser des éléments, forçant un recalcul de carte complet, qui peut échouer.\\
        \item Tous nos éléments restent valides, la carte préservant la grille peut être déposée directement sur le maillage initial.\\
        \centering
        \includegraphics[width=0.8\linewidth]{yoimg/restriction.png}
    \end{itemize}
\end{frame}

\begin{frame}{Résultats}
    \centering
    \includegraphics[width=\linewidth]{yoimg/5_meshes.png}
\end{frame}

\section{Contribution 3 : Maillage quadrilatère adapté à des déformations}
\begin{frame}
    \frametitle{Plan de la présentation}
    \tableofcontents[currentsection, sectionstyle=show/shaded, subsectionstyle=show/show/hide]
\end{frame}
%\subsection{Alternance entre maillage et simulation}
%\subsection{Valences adaptées aux géométries de déformations}
%\begin{frame}
%    \frametitle{Plan de la présentation}
%    \tableofcontents[currentsubsection, sectionstyle=show/shaded, subsectionstyle=show/shaded/hide]
%\end{frame}
\begin{frame}{Algorithm - Alternance maillage et simulation}
    \small
    \begin{algorithm}[H]
        \SetAlgoLined
        \label{algo:iterative_meshing_quadcover}
        \KwIn{Maillage de l'état initial de l'objet $Q_0$}
        \KwOut{Un maillage par pas de temps de la simulation $(Q_n)_{n \leq N}$}
        \SetKwProg{Fn}{Function}{:}{}
        \Fn{QuadcoverIteratif($Q_0$)}{
        $n_f \gets 0$\\
        \While{$n_f < N$}{
            Lancement de la simulation de déformation sur le maillage $Q_0$.\\
            $(Q_n)_{n \leq n_f} \gets$ un maillage quadrilatère par pas de temps réussi de la simulation\\
            \ForAll{$n \leq n_f$}{
                $T_n \gets Q_n$ divisé en un maillage triangulaire\\
            }
            $Q_0 \gets$ calcul d'un maillage initial adapté aux déformations $(T_n)_{n \leq N}$\\
        }
        \Return{$(Q_n)_{n \leq N}$}
        }
    \end{algorithm}
\end{frame}
\begin{frame}{Déterminer des valences de bord adaptés à la déformation}
    \small
    \begin{figure}
        \centering
        \includegraphics[width=0.46\linewidth]{img/quadsimu/coin_pb_0.PNG}
        \includegraphics[width=0.27\linewidth]{img/quadsimu/coin_pb_1.PNG}
        \caption{Les positions de singularités optimales pour une géométrie initiale peuvent conduire à des quadrilatères de qualité médiocre après une déformation.}
        \label{fig:asp_ratio_pb}
    \end{figure}
\end{frame}
\begin{frame}{Déterminer des valences de bord adaptés à la déformation}
    \small
    \begin{figure}
        \centering
        \includegraphics[width=0.39\linewidth]{img/quadsimu/coin_rs_0.PNG}
        \includegraphics[width=0.35\linewidth]{img/quadsimu/coin_rs_1.PNG}
        \caption{En utilisant toutes les géométries de l'itération précédente de la simulation échouée, 4 quadrilatères adjacents sont automatiquement placés au 
        point de bord où la déformation est la plus prononcée.}
        \label{fig:asp_ratio_sol}
    \end{figure}
\end{frame}

\begin{frame}{Carte sans couture adapté à la déformation}
    On fait la moyenne des $N$ champ de repère: 
    $$ \mu_i = \frac{1}{N} \sum_{n \leq N} \mu_{i, n} $$
    \pause
    On calcule une carte sans couture $u$ la plus proche possible de $\mu$:
    \begin{equation*} \label{eq:quadcover_energy}
        \begin{array}{ll}
            \underset{u}{\argmin} & \underset{t \in T}{\sum}\ \ \underset{i, j \in \mathcal{C}(t)}{\sum}  \left|\left| (u_i - u_j) - (\mu_i - \mu_j)  \right|\right|^2\\
            \text{sous contrainte: } & \begin{pmatrix} u_{i'} - u_{j'} \end{pmatrix} = R_{tt'} \begin{pmatrix}  u_{i} - u_{j} \end{pmatrix}. \\
        \end{array}
    \end{equation*}
    \pause
    Une carte sans couture est valide, si pour tout triangle $t$ et ses coins $i, j, k$:
    \begin{equation*}\label{eq:positive_jacobien_2D}
        \det \left(u_{j} - u_{i}, u_{k} - u_{i} \right) > 0
    \end{equation*}
\end{frame}

\begin{frame}{Résultats: Maillage quadrilatère non adapté vs adapté à la déformation}
    \begin{figure}
        \centering
        \includegraphics[width=0.54\linewidth]{img/quadsimu/deformation_same_step.PNG}
    \end{figure}
\end{frame}
 
\begin{frame}{Adaptation progressive à la déformation}
    \begin{figure}
        \centering
        \only<1>{\includegraphics[width=0.69\linewidth]{img/quadsimu/seal_simu_1.PNG}}
        \only<2>{\includegraphics[width=0.69\linewidth]{img/quadsimu/seal_simu_2.PNG}}
        \only<3>{\includegraphics[width=0.69\linewidth]{img/quadsimu/seal_simu_3.PNG}}
    \end{figure}
\end{frame}


\section{Conclusions, Perspectives}
\begin{frame}
    \frametitle{Plan de la présentation}
    \tableofcontents[currentsection, sectionstyle=show/shaded, subsectionstyle=show/show/hide]
\end{frame}
\begin{frame}{Travail réalisé : Maillage quadrilatère 2D}
    \centering
    \begin{enumerate}
        \item Une seule géométrie en entrée :
        \begin{itemize}
            \item Champ de repère adapté à des modèles CAO.
            \item Quantification sans recalcul de paramétrisation.
            %\item Implémentation web (ThreeJS + WebAssembly) : \url{ddesobry.github.io/quadmesher.html}
            \only<2-3>{
                \item Correction des problèmes de non-intégrabilité.    
                \item 100\% de réussite sur la b.d.d Mambo [\cite{ledoux_mambo_2019}]
            }
        \end{itemize}
        \only<3>{
            \item Maillage quad pour une simulation de déformation :
            \begin{itemize}
                \item Alternance automatisée entre simulation et maillage. 
                \item Valences de bord adaptées aux géométries de déformation.
                \item 3 simulations industrielles réalisées automatiquement. 
            \end{itemize}
        }
    \end{enumerate}
    \only<1>{
        \vspace{2em}
        \includegraphics[width=\linewidth]{img/quadsimu/5_meshes.png}
    }
    \only<2>{
        %\includegraphics[width=.49\linewidth]{img/quadsimu/solve_sharkanulus_2}
        \includegraphics[width=.7\linewidth]{img/quadsimu/sharkanulus_mambo}
    }
\end{frame}
    
\iffalse
\begin{frame}{Travail réalisé : Maillage hexaédrique 3D}
    \begin{enumerate}
        \item Réparation incrémentale des graphes de singularité
        \item Champ de repère avec singularités de bord imposées.
    \end{enumerate}

    \begin{itemize}
        \item "[\cite{ledoux_mambo_2019}]" / \cite{ray_practical_2016} / \cite{ray_practical_2016} + (1) / \cite{ray_practical_2016} + (2)
        \item "Basique" (74 modèles) / 18\% / 40\% / 69\%
        \item "Simple" (29 modèles) / 0\% / 21\% / 34\%
        \item "Medium" (9 modèles) / 0\% / 0\% / 0\%
    \end{itemize}
\end{frame}
\fi

\begin{frame}{Travail réalisé : Maillage hexaédrique 3D}
    \small
    \centering
    \begin{enumerate}
        \item Réparation de graphe de singularité de champ de repère \label{enum:reparation}
        \item Champ de repère avec singularités de bord imposées. \label{enum:champ}
    \end{enumerate}
    %L'algorithme de calcul de champ de repère est \cite{ray_practical_2016}
    \begin{table}
    \centering
    \scriptsize
    \begin{tabular}{|l|c|c|c|}
    \hline
    Base de données & [\cite{ray_practical_2016}] & + (\ref{enum:reparation}) & + (\ref{enum:champ}) \\
    \hline
    Mambo-Basique & 18\% & 40\% & 69\% \\
    \hline
    Mambo-Simple & 0\% & 21\% & 34\% \\
    \hline
    Mambo-Medium & 0\% & 0\% & 0\% \\
    \hline
    \end{tabular}
    \end{table}
    \% de maillages réussis  en fonction de l'utilisation de (\ref{enum:reparation}) et (\ref{enum:champ})
    \only<1>{
        \begin{columns}[T] % align columns
            \centering
            \begin{column}{.45\linewidth}
                \includegraphics[width=\linewidth]{img/hexmeshing_ff/resultats_3.png}
            \end{column}
            \begin{column}{.45\linewidth}
                \includegraphics[width=\linewidth]{img/hexmeshing_ff/prescribed_valences.png}
            \end{column}
        \end{columns}
    }
\end{frame}

\iffalse
\begin{frame}{Perspectives}
    \textbf{Maillage quadrilatère:}
    \begin{itemize}
        \item Correction champ de repère : Résoudre les problèmes de non-intégrabilité sans trop affecter la qualité.
        \item Intégrabilité champ de repère : Optimiser le placement des singularités sans trop affecter les performances.
    \end{itemize}
    \pause
    \textbf{Maillage hexaédrique:}
    \begin{itemize}
        \item Initialisation champ de repère : Méthode automatique pour déterminer des valences sur le bord valides.
        \item Correction champ de repère : Méthode de correction robuste des graphes de singularités intérieurs [\cite{liu2023locally}].
        \item Quantification rapide : Résoudre le problème de quantification par une méthode gloutonne comme en 2D.
    \end{itemize}
\end{frame}
\fi

\begin{frame}{Perspectives}
    \scriptsize
    \begin{table}
        \centering
        \begin{tabular}{|l|c|c|c|}
        \hline
        Base de données & [\cite{ray_practical_2016}] & + (\ref{enum:reparation}) & + (\ref{enum:champ}) \\
        \hline
        Mambo-Basique & 18\% & 40\% & 69\% \\
        \hline
        Mambo-Simple & 0\% & 21\% & 34\% \\
        \hline
        Mambo-Medium & 0\% & 0\% & 0\% \\
        \hline
        \end{tabular}
    \end{table}
    \small
    \textbf{Pistes d'amélioration de la paramétrisation globale 3D :}
    \begin{enumerate}
        \item Correction champ de repère : Méthode de correction robuste des graphes de singularités intérieurs [\cite{liu2023locally}].
        \item Initialisation champ de repère : Méthode automatique pour déterminer des valences sur le bord valides.
        \item Quantification 3D sans recalcul de paramétrisation : Décimation du maillage tétrahédrique d'entrée et résolution gloutonne.%Résoudre le problème de quantification sur un maillage tétraédrique décimé.
    \end{enumerate}
\end{frame}

\iffalse
\begin{frame}[allowframebreaks]{Bibliographie}
    \printbibliography
\end{frame}
\fi
% Slide de fin
\institute[Université de Lorraine] 
{
\centering
\vspace{.5cm}
\Large Merci pour votre attention
\vspace{.5cm}
\begin{center}
    \noindent
    \begin{minipage}{.193\textwidth}
        \centering
    \includegraphics[width=.8\linewidth]{img/new_images/inria.jpg}
    \end{minipage}%
    \begin{minipage}{.193\textwidth}
        \centering
        \includegraphics[width=.6\linewidth]{img/new_images/loria.png}
    \end{minipage}
    \begin{minipage}{.193\textwidth}
    \centering
    \includegraphics[width=.8\linewidth]{img/new_images/UL.png}
    \end{minipage}%
    \begin{minipage}{.193\textwidth}
        \centering
    \includegraphics[width=.6\linewidth]{img/new_images/total_energies.jpg}
    \end{minipage}
    \begin{minipage}{.193\textwidth}
        \includegraphics[width=.6\linewidth]{img/new_images/hutchinson.png}
    \end{minipage}
\end{center}
}
%\date{23 Août 2023}

%\frame{\titlepage}
\frame{\titlepage}
%\printbibliography
\end{document}