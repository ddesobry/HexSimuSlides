\begin{frame}
    \frametitle{Accomplissements de la thèse}
    \begin{itemize}
    \item 2D et 3D: Gain de contrôle sur les maillages produits grâce à l'utilisation de nouvelles méthodes de champ de repère.
    \item 2D: Méthode rapide (< 1 min) de génération de maillage quadrilatère de qualité sur des modèles industriels.
    \item 2D: Maillage quadrilatère adapté à plusieurs géométries de simulations de grandes déformations.
    \end{itemize}
\end{frame}
    
\begin{frame}
    \frametitle{Perspectives: Champ de repère intégrable et Réseau de singularité valide}
    \begin{itemize}
    \item Étudier les champs de direction \textit{intégrables} pour garantir l'existence d'une paramétrisation globale \textit{bijective}.
    \item Déterminer automatiquement des valences sur le bord pour un maillage hexaédrique et calculer un graphe de singularité intérieur valide.
    \end{itemize}
\end{frame}
    
\begin{frame}
    \frametitle{Perspectives: Arêtes caractéristiques et Quantification 3D}
    \begin{itemize}
    \item Explorer l'alignement d'une carte avec les arêtes caractéristiques lors de l'étape de quantification.
    \item Accélérer l'étape de quantification 3D en optimisant les positions des singularités dans la carte préservant la grille sur le maillage décimé 3D.
    \end{itemize}
\end{frame}
    