


\iffalse
\begin{frame}{Algorithm - Alternance maillage et simulation}
    \small
    \begin{algorithm}[H]
        \SetAlgoLined
        \label{algo:iterative_meshing_quadcover}
        \KwIn{Maillage de l'état initial de l'objet $Q_0$}
        \KwOut{Un maillage par pas de temps de la simulation $(Q_n)_{n \leq N}$}
        \SetKwProg{Fn}{Function}{:}{}
        \Fn{QuadcoverIteratif($Q_0$)}{
        $n_f \gets 0$\\
        \While{$n_f < N$}{
            Lancement de la simulation de déformation sur le maillage $Q_0$.\\
            $(Q_n)_{n \leq n_f} \gets$ un maillage quadrilatère par pas de temps réussi de la simulation\\
            \ForAll{$n \leq n_f$}{
                $T_n \gets Q_n$ divisé en un maillage triangulaire\\
            }
            $Q_0 \gets$ calcul d'un maillage initial adapté aux déformations $(T_n)_{n \leq N}$\\
        }
        \Return{$(Q_n)_{n \leq N}$}
        }
    \end{algorithm}
\end{frame}

\begin{frame}{Déterminer des valences de bord adaptées à la déformation}
    \small
    \begin{figure}
        \centering
        \includegraphics[width=0.46\linewidth]{img/quadsimu/coin_pb_0.PNG}
        \includegraphics[width=0.27\linewidth]{img/quadsimu/coin_pb_1.PNG}
        \caption{Les positions de singularités optimales pour une géométrie initiale peuvent conduire à des quadrilatères de mauvaise qualité après une déformation.}
        \label{fig:asp_ratio_pb}
    \end{figure}
\end{frame}
\fi
\begin{frame}{Problème: Valences inadaptées à une déformation}
    \begin{figure}
        \centering
        \includegraphics[width=\linewidth]{img/new_images/echec_simu.PNG}
        \caption{Les positions de singularités optimales pour une géométrie initiale peuvent conduire à des quadrilatères de mauvaise qualité après une déformation.}
    \end{figure}
\end{frame}

\begin{frame}{Adaptation d'un maillage quad à une simulation}
    \textbf{Alternance entre maillage et simulation} :\\
    \begin{enumerate}
        \item Calculer un maillage quad initial \( Q \)
        \item TANT QUE la simulation de déformation sur le maillage \( Q \) n'atteint pas l'itération finale $N$
        \begin{itemize}
            \item Récupérer les maillages quadrilatères déformés
            \item Recalculer un maillage \( Q \) adapté aux déformations récupérées
        \end{itemize}
    \end{enumerate}
    
\end{frame}

\begin{frame}{Déterminer des valences de bord adaptées à la déformation}
    \begin{columns}[c] % Le paramètre 'c' centre verticalement le contenu

        \column{0.5\textwidth}  % Largeur de la colonne de gauche
        \includegraphics[width=\linewidth]{img/new_images/evolution_verrin.PNG}

        \column{0.5\textwidth}  % Largeur de la colonne de droite
        \only<1>{\includegraphics[width=\linewidth]{img/new_images/simu_debut_verrin.PNG}}
        \only<2>{\includegraphics[width=\linewidth]{img/new_images/simu_milieu_verrin.PNG}}
        \only<3>{\includegraphics[width=\linewidth]{img/new_images/simu_fin_verrin.PNG}}
    \end{columns}
\end{frame}

\iffalse
\begin{frame}{Intégration à plusieurs géométries en entrée}
    \centering
    \only<1>{\textbf{Intégration de \blue{a} et \red{b} pour une géométrie}
    $$\argmin_{\blue{u}, \red{v}} \sum_{t, h} 
        \left\| \begin{pmatrix} \blue{u_{next(h)} - u_h} \\ \red{v_{next(h)} - v_h} \end{pmatrix}  - \begin{pmatrix} 
            < \blue{a_t}, \green{g_{next(h)} - g_h} > \\ < \red{b_t}, \green{g_{next(h)} - g_h} >\end{pmatrix} \right\|^2 $$}
    
    \only<2>{
        \textbf{Intégration de \blue{a} et \red{b} pour $N$ géométries}
    $$\argmin_{\blue{u}, \red{v}} \sum_{n\leq N} \sum_{t, h} 
        \left\| \begin{pmatrix} \blue{u_{next(h)} - u_h} \\ \red{v_{next(h)} - v_h} \end{pmatrix}  - \begin{pmatrix} 
            < \blue{a_{t, n}}, \green{g_{next(h), n} - g_{h, n}} > \\ < \red{b_{t, n}}, \green{g_{next(h), n} - g_{h, n}} >\end{pmatrix} \right\|^2 $$}

    \begin{columns}
        \begin{column}{0.5\textwidth}
            \begin{itemize}
            \item \( t \) : triangles du maillage
            \item \( h \) : demi-arêtes de \( t \)
            \item \( g \) : coordonnées (x, y)
            \end{itemize}
        \end{column}
    
        \begin{column}{0.5\textwidth}
            \includegraphics[width=.8\linewidth]{img/new_images/tri_g_u_v.PNG}
        \end{column}
    \end{columns}
\end{frame}
\begin{frame}{Carte sans couture adapté à la déformation}
    On fait la moyenne des $N$ champ de repère: 
    $$ \mu_i = \frac{1}{N} \sum_{n \leq N} \mu_{i, n} $$
    \pause
    On calcule une carte sans couture $u$ la plus proche possible de $\mu$:
    \begin{equation*} \label{eq:quadcover_energy}
        \begin{array}{ll}
            \underset{u}{\argmin} & \underset{t \in T}{\sum}\ \ \only<3>{\textcolor{red}{\lambda_t}} \underset{i, j \in \mathcal{C}(t)}{\sum} \left|\left| (u_i - u_j) - (\mu_i - \mu_j)  \right|\right|^2\\
            \text{sous contrainte: } & \begin{pmatrix} u_{i'} - u_{j'} \end{pmatrix} = R_{tt'} \begin{pmatrix}  u_{i} - u_{j} \end{pmatrix}. \\
        \end{array}
    \end{equation*}
    \pause
    Une carte sans couture est valide, si pour tout triangle $t$ et ses coins $i, j, k$:
    \begin{equation*}\label{eq:positive_jacobien_2D}
        \det \left(u_{j} - u_{i}, u_{k} - u_{i} \right) > 0
    \end{equation*}
\end{frame}
\fi

\begin{frame}{Maillage quadrilatère avant et après adaptation}
    \begin{columns}[T] % align columns
        \begin{column}{.5\textwidth}
        \vspace{1cm}
        \textbf{Avant adaptation:} \\
        Le joint traverse son support et la simulation échoue.
        
        \vspace{1.7cm}
        
        \textbf{Après adaptation:} \\
        Le joint se plie au niveau des points rouges de valence 3 et la simulation réussit.
        \end{column}%
        
        \begin{column}{.5\textwidth}
        \begin{figure}
            \centering
            %\includegraphics[width=\linewidth]{img/quadsimu/deformation_same_step.PNG}
            \includegraphics[width=\linewidth]{img/new_images/joint_with_closeups.PNG}
        \end{figure}
        \end{column}
    \end{columns}
\end{frame}
 \iffalse
\begin{frame}{Adaptation progressive à la déformation}
    \begin{figure}
        \centering
        \only<1>{\includegraphics[width=0.69\linewidth]{img/quadsimu/seal_simu_1.PNG}}
        \only<2>{\includegraphics[width=0.69\linewidth]{img/quadsimu/seal_simu_2.PNG}}
        \only<3>{\includegraphics[width=0.69\linewidth]{img/quadsimu/seal_simu_3.PNG}}
    \end{figure}
\end{frame}
\fi

\begin{frame}{Comparaison avec une autre méthode automatique}
    \begin{figure}
        \centering
        \includegraphics[width=0.99\textwidth]{img/quadsimu/comparison_with_bad_quality_mesh.PNG}
    \end{figure}
\end{frame}

\begin{frame}{Comparaison avec une méthode semi-manuelle}
    \centering
    \begin{figure}
        \centering
        \only<1>{\includegraphics[width=0.99\textwidth]{img/introduction/hutchinson_sapin.PNG}}
        \only<2>{\includegraphics[width=0.79\linewidth]{img/quadsimu/resistance_to_compression.PNG}}
	\end{figure}
    %\only<1>{3 semaines pour adapter manuellement le maillage}
    %\only<2>{2 heures d'alternance automatique maillage / simulation}
\end{frame}